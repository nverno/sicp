%Warning: This was Latex'ed on a Mac in order to include the figures
% this next is to generate figures on the Mac
\def\picture #1 by #2 (#3){
  $${\vbox to #2{
    \hrule width #1 height 0pt depth 0pt
    \vfill
    \special{picture #3} % this is the low-level interface
    }}$$
  }

\input  ../6001mac
%\input 6001mac

\begin{document}

\psetheader{Fall Semester, 1993}{Problem Set 4}

\medskip

\begin{flushleft}
Issued:  Tuesday, September 28 \\
\smallskip
Tutorial preparation for: Week of October 4\\
\smallskip
Written solutions due: Friday, October 8 in Recitation \\
\smallskip
Reading: 
\begin{tightlist}
\item Course notes: finish section 2.2
\item code files {\tt hend.scm} and {\tt hutils.scm} (attached to this handout)
\end{tightlist}
\end{flushleft}

\begin{center}
{\bf A Graphics Design Language}
\end{center}

In this assignment, you will work with the Peter Henderson's
``square-limit'' graphics design language, which Hal described in
lecture on September 28.  Before beginning work on this programming
assignment, you should review the notes for that lecture.  The goal of
this problem set is to reinforce ideas about data abstraction and
higher-order procedures, and to emphasize the expressive power that
derives from appropriate primitives, means of combination, and means
of abstraction.\footnote{This problem set was developed by Hal
Abelson, based upon work by Peter Henderson (``Functional Geometry,''
in {\em Proc. ACM Conference on Lisp and Functional Programming},
1982).  The image display code was designed and implemented by Daniel
Coore.}

\begin{figure}[b]
\picture 3.90 in by 1.81 in (escher)
\caption{{\protect\footnotesize
A picture of M.C. Escher, and the same picture transformed by the
square-limit process.}}
\label{escher}
\end{figure} 

Section 1 of this handout reviews the language, as presented in
lecture.  You will need to study this in order to prepare the
tutorial presentations in section 2.  Section 3 gives the lab assignment,
which includes an optional design contest.


\section{1. The Square-limit language}

Remember from lecture that the key idea in the square-limit language
is to use {\em painter} procedures that take frames as inputs and
paint images that are scaled to fit the frames.

\subsection{Basic data structures}

Vectors are represented as pairs of numbers.

\beginlisp
(define make-vect cons)
(define vector-xcor car)
(define vector-ycor cdr)
\endlisp

Here are the operations of vector addition, subtraction, and scaling a
vector by a number:

\beginlisp
(define (vector-add v1 v2)
  (make-vect (+ (xcor v1) (xcor v2))
             (+ (ycor v1) (ycor v2))))
\null
(define (vector-sub v1 v2)
  (vector-add v1 (vector-scale -1 v2)))
\null
(define (vector-scale x v)
  (make-vect (* x (xcor v))
             (* x (ycor v))))
\endlisp

A pair of vectors determines a directed line segment---the segment
running from the endpoint of the first vector to the endpoint of the
second vector:

\beginlisp
(define make-segment cons)
(define segment-start car)
(define segment-end cdr)
\endlisp

\subsubsection{Frames}

A frame is represented by three vectors: an origin and two edge
vectors.

\beginlisp
(define (make-frame origin edge1 edge2)
  (list 'frame origin edge1 edge2))
\null
(define frame-origin cadr)
(define frame-edge1 caddr)
(define frame-edge2 cadddr)
\endlisp

The frame's origin is given as a vector with respect to the origin of
the graphics-window coordinate system.  The edge vectors specify the
offsets of the corners of the frame from the origin of the frame.  If
the edges are perpendicular, the frame will be a rectangle; otherwise
it will be a more general parallelogram.  Figure~\ref{frames} shows a
frame and its associated vectors.

\begin{figure}
\picture 3.51 in by 2.42 in (frame)
\caption{{\protect\footnotesize A frame is represented by an origin
and two edge vectors.}}
\label{frames}
\end{figure} 

Each frame determines a system of ``frame coordinates'' $(x,y)$ where
$(0,0)$ is the origin of the frame, $x$ represents the displacement 
along the first edge (as a fraction of the length of the edge) and $y$ is the
displacement along the second edge.  For example, the origin of the
frame has frame coordinates $(0,0)$ and the vertex diagonally opposite
the origin has frame coordinates $(1,1)$.

Another way to express this idea is to say that each frame has an
associated {\em frame coordinate map} that transforms the frame
coordinates of a point into the Cartesian plane coordinates of the
point.  That is, $(x,y)$ gets mapped onto the Cartesian
coordinates of the point given by the vector sum
\begin{displaymath}
\hbox{\rm Origin(Frame)} + x\cdot \hbox{\rm Edge}_1\hbox{\rm (Frame)}
+ y\cdot \hbox{\rm Edge}_2\hbox{\rm (Frame)} 
\end{displaymath}

We can represent the frame coordinate map by the following procedure:\\
\beginlisp
(define (frame-coord-map frame)
  (lambda (point-in-frame-coords)
    (vector-add
     (frame-origin frame)
     (vector-add (vector-scale (vector-xcor point-in-frame-coords)
                               (frame-edge1 frame))
                 (vector-scale (vector-ycor point-in-frame-coords)
                               (frame-edge2 frame))))))
\endlisp

For example, {\tt ((frame-coord-map a-frame) (make-vect 0 0))} will
return the same value as {\tt (frame-origin a-frame)}.

The procedure {\tt make-relative-frame} provides a convenient way to
transform frames.  Given a frame and three points {\tt origin}, {\tt
corner1}, and {\tt corner2} (expressed in frame coordinates), it
returns a new frame with those corners:

\beginlisp
(define (make-relative-frame origin corner1 corner2)
  (lambda (frame)
    (let ((m (frame-coord-map frame)))
      (let ((new-origin (m origin)))
        (make-frame new-origin
                    (vector-sub (m corner1) new-origin)
                    (vector-sub (m corner2) new-origin))))))
\endlisp

\noindent
For example,\\
\beginlisp
(make-frame-relative (make-vect .5 .5) (make-vect 1 .5) (make-vect .5 1))
\endlisp

\noindent
returns the procedure that transforms a frame into the upper-right
quarter of the frame.

\subsubsection{Painters}

As described in lecture, a painter is a procedure that, given a frame
as argument, ``paints'' a picture in the frame.  That is to say, if
{\tt p} is a painter and {\tt f} is a frame, then evaluating {\tt (p
f)} will cause an image to appear in the frame.  The image will be
scaled and stretched to fit the frame.

The language you will be working with includes four ways to create
primitive painters.

The simplest painters are created with {\tt number->painter}, which
takes a number as argument.  These painters fill a frame with a solid
shade of gray.  The number specifies a gray level: 0 is black, 255 is
white, and numbers in between are increasingly lighter shades of
gray.  Here are some examples:

\beginlisp
(define black (number->painter 0))
(define white (number->painter 255))
(define gray (number->painter 150))
\endlisp

\noindent
You can also specify a painter using {\tt procedure->painter}, which
takes a procedure as argument.  The procedure determines a gray level
(0 to 255) as a function of $(x,y)$ position, for example:

\beginlisp
(define diagonal-shading
  (procedure->painter (lambda (x y) (* 100 (+ x y)))))
\endlisp

\noindent
The $x$ and $y$ coordinates run from 0 to 1 and specify the fraction
that each point is offset from the frame's origin along the frame's
edges.  (See figure~\ref{frames}.) Thus, the frame is filled out by
the set of points $(x,y)$ such that $0\leq x,y \leq 1$.

A third kind of painter is created by {\tt segments->painter}, which
takes a list of line segments as argument.  This paints the line
drawing specified by the list segments.  The $(x,y)$ coordinates of
the line segments are specified as above.  For example, you can make
the ``Z'' shape shown in figure~\ref{primitive-painters} as

\beginlisp
(define mark-of-zorro
  (let ((v1 (make-vect .1 .9))
        (v2 (make-vect .8 .9))
        (v3 (make-vect .1 .2))
        (v4 (make-vect .9 .3)))
    (segments->painter
     (list (make-segment v1 v2)
           (make-segment v2 v3)
           (make-segment v3 v4)))))
\endlisp

The final way to create a primitive painter is from a stored image.
The procedure {\tt load-painter} uses an image from the 6001 image
collection to create a painter.\footnote{The images are kept in the
directory {\tt 6001-images}.  Use the Edwin command
{\tt M-x list-directory} to see entire contents of the directory.
Each image is $128\times 128$, stored in ``pgm'' format.}
or instance:

\beginlisp
(define fovnder (load-painter "fovnder"))
\endlisp

\noindent
will paint an image of William Barton Rogers, the {\sc fovnder} of MIT.
(See figure~\ref{primitive-painters}.) 

\begin{figure}
\picture 3.89 in by 1.82 in (primpics)
\caption{{\protect\footnotesize
Examples of primitive painters: {\tt mark-of-zorro} and {\tt fovnder}.}}
\label{primitive-painters}
\end{figure} 

\subsection{Transforming and combining painters}

We can transform a painter to produce a new painter which, when
given a frame, calls the original painter on the transformed frame.
For example, if {\tt p} is a painter and {\tt f} is a frame, then\\
\beginlisp
(p ((make-frame-relative (make-vect .5 .5) (make-vect 1 .5) (make-vect .5 1))
    f))
\endlisp

\noindent
will paint in the upper-right-hand corner of the frame.

We can abstract this idea with the following procedure:

\beginlisp
(define (transform-painter origin corner1 corner2)
  (lambda (painter)
    (compose painter
             (make-relative-frame origin corner1 corner2))))
\endlisp

\noindent
Calling this with an origin and two corners, returns a procedure that
transforms a painter into one that paints relative to a new frame with
the specified origin and corners.  For example, we could define:\\
\beginlisp
(define (shrink-to-upper-left painter)
  ((transform-painter (make-vect .5 .5) (make-vect 1 .5) (make-vect .5 1))
   painter))
\endlisp

\noindent
Note that this can be written equivalently as\\
\beginlisp
(define shrink-to-upper-left
  (transform-painter (make-vect .5 .5) (make-vect 1 .5) (make-vect .5 1)))
\endlisp

Other transformed frames will flip images horizontally:\\
\beginlisp
(define flip-horiz
  (transform-painter (make-vect 1 0)
                     (make-vect 0 0)
                     (make-vect 1 1)))
\endlisp

\noindent
or rotate images counterclockwise by 90 degrees:\\
\beginlisp
(define rotate90
  (transform-painter (make-vect 1 0)
                     (make-vect 1 1)
                     (make-vect 0 0)))
\endlisp

By repeating rotations, we can create painters whose images are
rotated through 180 or 270 degrees:

\beginlisp
(define rotate180 (repeated rotate90 2))
(define rotate270 (repeated rotate90 3))
\endlisp

We can combine the results of two painters a single frame by
calling each painter on the frame:

\beginlisp
(define (superpose painter1 painter2)
  (lambda (frame)
    (painter1 frame)
    (painter2 frame)))
\endlisp

To draw one image beside another, we combine one in the left half
of the frame with one in the right half of the frame:

\beginlisp
(define (beside painter1 painter2)
  (let ((split-point (make-vect .5 0)))
    (superpose
     ((transform-painter zero-vector
                         split-point
                         (make-vect 0 1))
      painter1)
     ((transform-painter split-point
                         (make-vect 1 0)
                         (make-vect .5 1))
      painter2))))

\endlisp

We can also define painters that combine painters vertically, by
using {\tt rotate} together with {\tt beside}.  The painter produced
by {\tt below} shows the image for {\tt painter1} below the image
for {\tt painter2}:

\beginlisp
(define (below painter1 painter2)
  (rotate270 (beside (rotate90 painter2)
                     (rotate90 painter1))))
\endlisp

\section{2. Tutorial exercises}

You should prepare the following exercises for oral presentation in
tutorial.  They cover material in sections 2.1 and 2.2 of the text,
and they also test your understanding of the square-limit language
described above, in preparation for doing the lab in section 1 of
this handout.  You may wish to use the computer to check your answers
to these questions, but you should try to do them (at least
initially) without the computer.

\paragraph{Tutorial exercise 1:}
Show how to define a procedure {\tt butlast} that, given a list,
returns a new list containing all but the last element of the original
list.  For example

\beginlisp
(butlast '(1 2 3 4 5 6))
;Value: (1 2 3 4 5)
\endlisp

\paragraph{Tutorial exercise 2:}
In the square-limit language, a frame is represented as a list of four
things---the symbol {\tt frame} followed by the origin and the two
edge vectors.

\begin{enumerate}

\item Pick some values for the coordinates of origin and edge vectors
and draw the box-and-pointer structure for the resulting frame.

\item Suppose we change the representation of frames and represent
them instead as a list of three vectors---the origin and the two
edges---without including the symbol {\tt frame}.  Give the new
definitions of {\tt make-frame}, {\tt frame-origin}, {\tt
frame-edge1}, and {\tt frame-edge2} for this representation.  In
addition to changing these constructors and selectors, what other
changes to the implementation of the square-limit language are
required in order to use this new representation?

\item Why might it be useful to include the symbol {\tt frame} as part
of the representation of frames?
\end{enumerate}

\paragraph{Tutorial exercise 3:}
Describe the patterns drawn by

\beginlisp
(procedure->painter (lambda (x y) (* x y)))
(procedure->painter (lambda (x y) (* 255 x y)))
\endlisp

\paragraph{Tutorial exercise 4:}
Captain Abstraction is insulted that the Mark of Zorro appears in this
problem set, while his mark does not.  Show how to use {\tt
segments->painter} to draw an ``A'' for Captain Abstraction.  Don't
worry about getting the coordinates exactly correct---just get the
general shape right.

\paragraph{Tutorial exercise 5:}
Section 1 defines {\tt below} in terms of {\tt beside} and {\tt
rotate}.  Give an alternative definition of {\tt below} that does not
use {\tt beside}.

\paragraph{Tutorial exercise 6:}
Define the painter transformation {\tt flip-vertically}.

\paragraph{Tutorial exercise 7:}
Describe the effect of

\beginlisp
(transform-painter (make-vect .1 .9)
                   (make-vect 1.5 1)
                   (make-vect .2 0))
\endlisp

\section{3. To do in lab}

Load the code for problem set 4, which contains the procedures
described in section 1.  You will not need to modify any of these.  We
suggest that you define your new procedures in a separate (initially
empty) editor buffer, to make it easy to reload the system if things
get fouled up.

When the problem set code is loaded, it will create three graphics windows,
named {\tt g1}, {\tt g2}, and {\tt g3}.  To paint a picture in a
window, use the procedure {\tt paint}.  For example,

\beginlisp
(paint g1 fovnder)
\endlisp

\noindent
will show a picture of William Barton Rogers in window {\tt g1}.
There is also a procedure called {\tt paint-hi-res}, which paints the
images at higher resolution ($256 \times 256$ rather than $128 \times
128$).  Painting at a higher resolution produces better looking
images, but takes four times as long.  As you work on this problem
set, you should look at the images using {\tt paint}, and reserve {\tt
paint-hi-res} to see the details of images that you find
interesting.\footnote{Painting a primitive image like {\tt fovnder} won't
look any different at high resolution, because the original picture is
only $128 \times 128$.  But as you start stretching and shrinking the
image, you will see differences at higher resolution.}

\paragraph{Printing pictures} You can print images on the laserjet
printer with Edwin's {\tt M-x print-graphics} command as described
in the {\em Don't Panic} manual.  The laserjet cannot print true
greyscale pictures, so the pictures will not look as good as they do
on the screen. Please print only a few images---only the ones that you
really want---so as not to waste paper and clog the printer queues.
We suggest that you print only images created with {\tt paint-hi-res},
not {\tt paint}.

\paragraph{Lab exercise 1:}
Make a collection of primitive painters to use in the rest of this
lab.  In addition to the ones predefined for you (and listed in
section 1), define at least one new painter of each of the four
primitive types: a uniform grey level made with {\tt number->painter},
something defined with {\tt procedure->painter}, a line-drawing made
with {\tt segments->painter}, and an image of your choice that is
loaded from the 6001 image collection with {\tt load-painter}.  Turn
in a list of your definitions.

\paragraph{Lab exercise 2:}
Experiment with some combinations of your primitive painters, using
{\tt beside}, {\tt below}, {\tt superpose}, flips, and rotations, to get a
feel for how these means of combination work.  You needn't turn in
anything for this exercise.

\paragraph{Lab exercise 3}
The ``diamond'' of a frame is defined to be the smaller frame
created by joining the midpoints of the original frame's sides, as shown in
figure~\ref{diamond}a.  Define a procedure {\tt
diamond} that transforms a painter into one that paints its image in
the diamond of the specified frame, as shown in
figure~\ref{diamond}b.  Try some examples, and turn in a listing of
your procedure.

\begin{figure}
\picture 3.72 in by 1.82 in (diamond)
\caption{{\protect\footnotesize
(a) The ``diamond of a frame is formed by joining the midpoints
of the sides.  (b) Painting created by {\tt (diamond fovnder)}.}}
\label{diamond}
\end{figure} 

\paragraph{Lab exercise 4}

The ``diamond'' transformation has the property that, if you start
with a square frame, the diamond frame is still square
(although rotated).  Define a transformation similar to {\tt
diamond}, but which does not produce square frames.  Try your
transformation on some images to get some nice effects.  Turn in a
listing of your procedure.

\paragraph{Lab exercise 5}
The following recursive {\tt right-split} procedure was demonstrated
in lecture:

\beginlisp
(define (right-split painter n)
  (if (= n 0)
      painter
      (let ((smaller (right-split painter (- n 1))))
        (beside painter (below smaller smaller)))))
\endlisp

Try this with some of the painters you've previously defined, both
primitives and combined ones.  Now define an analogous {\tt up-split}
procedure as shown in figure~\ref{up-split}.  Make sure to test it on
a variety of painters.  Turn in a listing of your procedure.  (In
defining your procedure, remember that {\tt (below painter1
painter2)} produces {\tt painter1} below {\tt painter2}.)

\begin{figure}
\picture 3.75 in by 1.79 in (up-split)
\beginlisp
                  (up-split mark-of-zorro 4)      (up-split fovnder 4)
\endlisp
\caption{{\protect\footnotesize
The {\tt up-split} procedure places a picture below two (recursively)
up-split copies of itself.}}
\label{up-split}
\end{figure} 


\paragraph{Lab exercise 6}
{\tt Right-split} and {\tt up-split} are both examples of a common
pattern that begins with a means of combining two painters and applies
this over and over in a recursive pattern.  We can capture this idea
in a procedure called {\tt keep-combining}, which takes as argument a
{\tt combiner} (which combines two painters).  For instance, we should
be able to give an alternative definition of {\tt right-split} as

\beginlisp
(define new-right-split
  (keep-combining
   (lambda (p1 p2)
     (beside p1 (below p2 p2)))))
\endlisp

\noindent
Complete the following definition of {\tt keep-combining}:

\beginlisp
(define (keep-combining combine-2)
  ;; combine-2 = (lambda (painter1 painter2) ...)
  (lambda (painter n)
    ((repeated
      {\rm $\langle$ fill in missing expression $\rangle$}
      n)
     painter)))
\endlisp

\noindent
Show that you can indeed define {\tt right-split} using your
procedure, and give an analogous definition of {\tt up-split}.


\paragraph{Lab exercise 7}
Once you have {\tt keep-combining}, you can use it to define lots of
recursive means of combination.  Figure~\ref{keep-combining} shows
three examples.  Invent some variations of your own.  Turn in the code
and one or two sample pictures.

\begin{figure}
\picture 1.81 in by 1.81 in (nest-fovnder)

\beginlisp
                     (define nest-diamonds
                       (keep-combining
                        (lambda (p1 p2) (superpose p1 (diamond p2)))))
\null
                     (nest-diamonds fovnder 4)
\endlisp

\picture 4.42 in by 1.81 in (combine)
\beginlisp
     (define new-comb                              (define mix-with-fovnder
       (keep-combining                               (keep-combining
        (lambda (p1 p2)                               (lambda (p1 p2)
          (square-limit (below p1 p2) 2))))             (below (beside p1
                                                                       fovnder)
                                                               (beside p2 p2)))))
\null                                                      
     (new-comb mark-of-zorro 2)                     (mix-with-fovnder romana 3)
\endlisp
\caption{{\protect\footnotesize
Some recursive combination schemes, defined with {\tt keep-combining}.}}
\label{keep-combining}
\end{figure} 

\paragraph{Lab exercise 8}
Define the {\tt square-limit} transformation illustrated in
figure~\ref{escher}.  The recursive plan for this transformation was
described in lecture.  Using {\tt right-split} and {\tt up-split},
define a {\tt corner-split} transformation that returns a painter
that will generate one-fourth of the desired image.  Then apply a
transformation {\tt 4times} to the {\tt corner-split} painter.  {\tt
4times} returns a painter that paints its image (suitably rotated) in
each quarter of its argument frame.  Turn in a listing of the
procedures you define: {\tt corner-split}, {\tt 4times}, and any
others.  Also turn in a clear explanation (a few sentences together
with a diagram) of the recursive plan for {\tt corner-split}.


\paragraph{Lab exercise 9}
The procedures you have implemented give you a wide choice of things
to experiment with.  Invent some new means of combination, both simple
ones like {\tt beside} and complex higher-order ones like {\tt
keep-combining} and see what kinds of interesting images you can create.
Turn in the code and one or two figures.

\paragraph{Contest (Optional)}

Hopefully, you generated some interesting designs in doing this
assignment.  If you wish, you can enter printouts of your best designs
in the 6.001 PS4 design contest.  Turn in your design collection
together with your homework, but {\em stapled separately}, and make
sure your name is on the work.  For each design, show the expression
you used to generate it.  Designs will be judged by the 6.001 staff and
other paragons of design expertise, and fabulous prizes will be
awarded in lecture.  There is a limit of five entries per student.
Make sure to turn in not only the pictures, but also the procedure(s) that
generated them.


\end{document}

